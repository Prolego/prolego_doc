\begin{resumo}
A robótica está cada vez mais inserida no âmbito educacional, desde a escola até a universidade, proporcionando aos alunos uma experiência única de aprendizagem. Com o lançamento do kit LEGO Mindstorms, a robótica ficou cada vez mais interessante aos olhos dos jovens, pois montar um robô com sensores, motores e programá-lo ficou mais acessível. Ao se popularizar, o kit LEGO Mindstorms tornou-se um dos principais integrantes dos torneios de robótica, até chegar ao ponto de existirem torneios somente para ele, como o Torneio de Robótica FIRST LEGO League. Nesses torneios, a maior preocupação é conseguir ganhar a maior quantidade de pontos possíveis em um espaço de tempo relativamente curto, sendo necessária a elaboração de estratégias para otimizar o gasto do tempo. O objetivo desse trabalho é a criação de uma máquina de raciocínio para otimizar o ganho de pontos, durante a quantidade de tempo estabelecida, priorizando, dentre as missões não realizadas, quais devem ser executadas pelo robô. Este trabalho será \textit{open-source}, desenvolvido na linguagem de programação Prolog, e testado no robô LEGO Mindstorms NXT, por meio de uma conexão \textit{bluetooth}.

\vspace{\onelineskip}
    
 \noindent
 \textbf{Palavras-chaves}: robótica educacional. máquina de raciocínio. prolog. programação dinâmica. NXT. LEGO. 
\end{resumo}
