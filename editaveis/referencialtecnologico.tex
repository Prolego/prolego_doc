\chapter{Referencial Tecnológico}

\section{Robótica Educacional}
\subsection{Kit Educacional Lego Mindstorms}
A LEGO lançou em janeiro de 2006, na feira Consumer Electronics Show em Las Vegas a linha Mindstorms NXT, que é uma versão mais avançada que a já consagrada RCX, possuindo um processador Atmel ARM 32 bits com clock de 48MHz, HD de 256KB de memória flash, memória RAM de 64KB, software próprio, sensores de luz, toque e som que permite que o robô tenha noções de distância, seja capaz se reagir a movimentos, ruídos e cores, e executa movimentos com um melhor grau de precisão que seu antecessor.

FALTA COLOCAR O QUE SERÁ UTILIZADO PARA RODAR O CÓDIGO EM PROLOG NO ROBÔ.

\section{Engenharia de Software}
\subsection{Ubuntu}
O Linux foi criado por Linus Torvalds em 1991 e desde então amplamente aderido pela comunidade. Utilizando esse kernel vários outros sistemas operacionais foram criados, dentre eles o Ubuntu. 
O Ubuntu foi criado pela comunidade Linux, baseado no Debian e patrocinado pela Canonical Ltd. e é licenciado por General License Public (GPL) podendo ser então modificado e distribuído de acordo com os termos da licença. Robustez e confiabilidade, por meio de código aberto, compatibilidade, segurança, rapidez e robustez são algumas características citadas pelos criadores. 

\subsection{SWIProlog}
O SWIProlog, criado em 1986 por Jan Wielemaker, é uma implementação, de código aberto, para a linguagem de programação Prolog, executando em modo texto, através de comandos 	no terminal do sistema. Sob a licença Lesser GNU Public License, pode ser utilizado nas plataformas Windows, Linux e MacOS. Possui diversas ferramentas de edição gráfica, tais como, J-Prolog Editor e SWI-Prolog-Editor.
Permite a utilização da linguagem Prolog por outras linguagens, tais como, C/C++ e Java.

\subsection{Bizagi Process Modeler}
O BizAgi Process Modeler é um aplicativo gratuito utilizado para criar e documentar modelos de processos em BPMN. Tal, faz parte de uma suíte de software, chamada Bizagi, composta por dois produtos: O Bizagi Process Modeler e o BizAgi BPM Suite. 

\subsection{Git}
O Git é um dos mais consagrados e utilizados sistemas de controle de versão distribuída. É open-source e gratuito, distribuído sob a licença GNU GPLv2, e projetado para lidar com qualquer tamanho de projeto mantendo a rapidez e eficiência. 

\subsection{Github}
O GitHub é um repositório web planejado para utilizadores do sistema de controle de versão Git. Possui opções de utilização de repositórios privados e públicos, sendo o pago e gratuito, respectivamente.
No GitHub é possível, dentre tantas outras funcionalidades, visualizar o código no navegador, criar \textit{issues}, revisar e aceitar mudanças, baixar o repositório como \textit{.zip}, criar \textit{branchs} ou até mesmo fazer o fork de um outro repositório.

\subsection{Waffle}
O Waffle é uma solução de gerenciamento de projeto online, que se integra ao repositório do GitHub e cria um board, muito semelhante a um kanban, com as issues que estão cadastradas, e as agrupa de acordo com seu status, que pode ser Backlog, Ready, In Progress  e Done. Cada issue, se estiver associada a um Milestone no GitHub, será também associada a um Milestone no Waffle. Qualquer alteração feita no GitHub é automaticamente alterado no Waffle.

\subsection{LaTeX}
O Latex é um sistema de preparação de documentos para composição tipográfica de alta qualidade, desenvolvido inicialmente por Leslie Lamport, em 1985, baseado na linguagem TeX criada por Donald E. Knuth, no final da década de 70, na Universidade de Stanford. Atualmente o Latex é mantido e desenvolvido pelo Projeto Latex3, e é disponibilizado gratuitamente.
O Latex é mais amplamente utilizado no meio técnico-científico para escrita de documentos de médio a grande porte, devido a sua facilidade de produzir fórmulas e símbolos matemáticos. 

\subsection{TexMaker}
O TeXMaker é um editor de texto Latex gratuito, multiplataforma para Linux e MacOSX, com suporte a unicode, verificação ortográfica, auto-completar e contém um visualizador de pdf embutido.
Com uma interface simples, o TexMaker contém botões de atalho para as funções mais utilizadas como: estruturas (part/chapter/section), referências (ref/cite), tamanho e estilo de letra, negrito, itálico, alinhar (direta/esquerda,centro) e inserir diversos simbolos matemáticos.

\section{Considerações finais}