\begin{resumo}[Abstract]
 \begin{otherlanguage*}{english}
   The robotics is more and more inserted in the education extent, since the school until the university, providing to the pupils a single learning experience. With the launch of the kit LEGO Mindstorms, the robotics is more and more interesting to the young people's eyes, because to mount a robot with sensors, motors and to plan it is more accessible. With its popularizing, the kit LEGO Mindstorms became one of the main integrants of the tournaments of robotics, until reaching the point of create tournaments only for it, like the Tournament of Robotics FIRST LEGO League. In these tournaments, the biggest preoccupation is to win the biggest quantity of possible points in short time, and it's necessary the preparation of strategies to optimize the waste of the time. Through the dynamic programming, the objective of this work is the creation of a machine of reasoning to optimize to the profit of points during the established quantity of time, prioritizing, among the missions that weren't realized, those that must be executed by a robot. This work will be open-source, developed in the language of programming Prolog, and tested in the robot LEGO Mindstorms NXT, through a connection bluetooth.

   \vspace{\onelineskip}
 
   \noindent 
   \textbf{Key-words}: educational robotics. reasoning engine. prolog. dynamic programming. NXT. LEGO.
 \end{otherlanguage*}
\end{resumo}
