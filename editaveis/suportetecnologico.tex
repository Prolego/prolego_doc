\chapter{Suporte Tecnológico}
Para dar suporte ao desenvolvimento do projeto foi feito um levantamento de ferramentas candidatas, sendo este refinado e descrito a seguir.

\section{Robótica Educacional}
Um dos maiores precursores da robótica no âmbito educacional foi Seymour Papert ao criar a  linguagem LOGO com o intuito de incentivar a aprendizagem da matemática, baseado nas idéias do suiço Jean Piaget que dizia, \textit{"as funções essenciais da inteligência consistem em compreender e inventar, em outras palavras, construir estruturas estruturando o real”}, ou seja, é essencial para a formação da inteligência a ação sobre objetos, e por meio dessa descobrir propriedades por meio de abstração.
Desde então, a robótica educacional vem se tornando uma plataforma atraente para criar envolvimento nos estudantes, incentivando o estudo da ciência e da tecnologia. 
As plataformas iterativas/lúdicas vem ganhando espaço ao longo dos anos como ferramentas auxiliares às metodologias de ensino que utilizam a robótica educacional. Dentre essas plataformas podemos citar: Robomind\footnote{www.robomindacademy.com}, Scratch\footnote{www.scratch.mit.edu}, Arduino\footnote{www.arduino.cc}, Raspberry Pi\footnote{www.raspberrypi.org} e o Kit Lego Mindstorms\footnote{mindstorms.lego.com}. Essas plataformas serão brevemente descritas a seguir.

\subsection{Robomind}
O Robomind é uma IDE para a programação dos movimentos de um robô em um mundo bidimensional, através de uma linguagem de programação bem simples e intuitiva, ideal para iniciantes. 
A interface da IDE é composta por quatro partes: à cima contêm o menu e os atalhos para as funções mais utilizadas, como inserir qualquer movimento; à esquerda encontra-se a área de  edição do código; à direita contém o mundo bidimensional, onde é possível visualizar o robô realizando os movimentos, previamente determinados pelo código, em um mapa; e na parte inferior encontra-se o controle de execução do programa juntamente à área de mensagens utilizadas pela IDE para informar erros sintáticos ou durante a execução do código, e a situação na qual o robô se encontra. 
O Robomind é gratuito para teste durante 30 dias. Após esse periodo é possível compra-lo por USD 14.00 por ano. Ele pode ser instalado em vários idiomas, dentre eles o Português\footnote{www.robomind.net/pt/}, além de ser multiplataforma, Linux, Mac e Windows. 
Com o Robomind só é possível simular o comportamento do que é implementado, não trabalhando a montagem de um robô, programação embarcada e utilização de sensores.

\subsection{Scratch}
O Scratch é um ambiente e uma linguagem de programação multimídia, produzido pelo \textit{MediaLab}\footnote{www.media.mit.edu} do MIT, para ser utilizado por crianças a partir dos oito anos de idade para criação de histórias iterativas, animações, jogos, músicas, dentre outros, ditos projetos Scratch. 
Um projeto Scratch contém personagens programáveis, que podem tratar eventos vindos do teclado ou do mouse, para mudança de posição e direção, dentre outras funcionalidades.
Assim como o Robomind, ele utiliza uma linguagem de programação em blocos, com encaixe seletivo ideal para programadores iniciantes. O Scratch é gratuito, \textit{online} e também tem a opção de utilizá-lo em português.
Similar ao Robomind, no Scratch só é possível simular de forma limitada a implementação, não sendo possível testar em um robô físico nem utilizar sensores.

\subsection{Arduino}
O Arduino é uma plataforma de prototipagem eletrônica \textit{open-source}, criada em 2005 pelo italiano Massimo Banzi, para auxiliar no ensino de eletrônica visando o baixo custo para os alunos. A plataforma é composta por um \textit{hardware} e um \textit{software} bastante flexíveis. 
O \textit{hardware} é constituído de uma placa com um microprocessador; porta USB, usada para comunicação serial com o computador; pinos digitais, utilizados para detecção ou transmissão de controles digitais; pinos analógicos, usados para leitura de sinais de sensores; e pinos de alimentação, usados para alimentação de circuitos externos. 
O \textit{software} é uma IDE, onde será programado o código, conhecido como \textit{sketch}, e por meio desta será passado à placa através de uma comunicação serial. Nesta IDE é utilizada a linguagem de programação Arduino, mas quando é passado para a placa, esta linguagem é traduzida para a linguagem estruturada C. Sua IDE é multiplataforma, podendo ser instalada em Linux, Mac e Windows.
O Arduino tem a vantagem de utilizar programação embarcada no robô e possuir entradas para diversos sensores, porém a montagem do robô fica a critério do desenvolvedor não tendo padronização alguma.

\subsection{Raspberry Pi}
O Raspberry Pi é um computador versátil e flexível, criado pela \textit{Raspiberry Pi Foundation}\footnote{www.raspberrypi.org/about/} em conjunto com a Universidade de Cambrigde, no ano de 2012, com o objetivo de estimular o ensino da ciência da computação em jovens do ensino básico.
O Raspberry Pi possui dois modelos, ambos equipados com processador multimídia \textit{Broadcom} BCM2835 \textit{system-on-chip} (SoC) de 700 Mhz com placa gráfica integrada VideoCore IV; entrada para cartão de memória, que se faz necessário pois o Raspberry Pi não possui armazenamento interno; interface HDMI; entrada USB e RCA; processador ARM 7 com capacidade de processamento de 32 bits, não sendo possível a instalação do Windows, mas aceitando qualquer distribuição Linux, como o Raspbian que é baseado no sistema operacional Debian. O que difere nas duas versões é a memoria RAM. Um modelo é oferecido com 256 MB e o outro com 512 MB.
Assim como o Arduino, a desvantagem do Raspberry Pi é a não padronização da montagem dos robôs. Entretanto, é possível a utilização de sensores e programação embarcada.

\subsection{Kit Educacional Lego Mindstorms}
A LEGO lançou em janeiro de 2006, na feira \textit{Consumer Electronics Show}\footnote{www.cesweb.org} em Las Vegas, a linha Mindstorms NXT, que é uma versão mais avançada que a já consagrada RCX, possuindo um processador Atmel ARM 32 bits com \textit{clock} de 48MHz, HD de 256KB de memória \textit{flash}, memória RAM de 64KB, software próprio, sensores de luz, toque e som. Esse suportr confere ao robô noções de distância, sendo inclusive capaz de reagir a movimentos, ruídos e cores. Essa nova versão também executa movimentos com maior grau de precisão que seu antecessor.
O kit Lego Mindstorms é bem flexível quanto a montagem do robô, porém a Lego disponibiliza um manual de montagem de um robô padrão, uma vantagem ao se comparar com os concorrentes mencionados anteriormente. 
A escolha do kit Lego Mindstorms para o dado projeto se deu, além da vantagem já conhecida da padronização do robô, por ele já conter os sensores necessários no kit e por esse material ser utilizado na matéria Princípios de Robótica Educacional, ministrada na FGA pelo prof. Dr. Maurício Serrano, co-orientador deste projeto.  

FALTA COLOCAR O QUE SERÁ UTILIZADO PARA RODAR O CÓDIGO EM PROLOG NO ROBÔ.


\section{Engenharia de Software}
\subsection{Ubuntu}
O Linux foi criado por Linus Torvalds em 1991 \cite{torvalds2001just} e desde então amplamente aderido pela comunidade. Com base nesse kernel vários outros sistemas operacionais foram criados, dentre eles o Ubuntu. 
Baseado no Debian e patrocinado pela Canonical Ltda., o Ubuntu é licenciado por General License Public (GPL), podendo ser então modificado e distribuído de acordo com os termos da licença. Robustez e confiabilidade, por meio de código aberto, compatibilidade, segurança, rapidez e robustez são algumas características que o fazem tão popular. 

\subsection{SWIProlog}
O SWIProlog é uma implementação, de código aberto, para a linguagem de programação Prolog, executando em modo texto, através de comandos no terminal do sistema. Sob a licença \textit{Lesser GNU Public License}\footnote{http://www.gnu.org/licenses/lgpl.html}, pode ser utilizado nas plataformas Windows, Linux e MacOS. Possui diversas ferramentas de edição gráfica, tais como: J-Prolog Editor e SWI-Prolog-Editor.
Permite ainda a utilização da linguagem Prolog por outras linguagens, tais como: C/C++ e Java.

\subsection{Bizagi Process Modeler}
O BizAgi Process Modeler\footnote{http://www.bizagi.com/en/bpm-suite/bpm-products/modeler} é um aplicativo gratuito utilizado para criar e documentar modelos de processos em BPMN\footnote{www.bpmn.org} (\textit{Business Process Model and Notation}). Esse aplicativo faz parte de uma suíte de software, chamada Bizagi, composta por dois produtos: o Bizagi Process Modeler e o BizAgi BPM Suite. 

\subsection{Git}
O Git\footnote{http://www.git-scm.com/} é um dos mais consagrados e utilizados sistemas de controle de versão distribuída. É \textit{open-source} e gratuito, distribuído sob a licença GNU GPLv2\footnote{http://www.gnu.org/licenses/gpl-2.0.html}, e projetado para lidar com qualquer tamanho de projeto mantendo a rapidez e eficiência. 

\subsection{Github}
O GitHub é um repositório web planejado para utilizadores do sistema de controle de versão Git. Possui opções de utilização de repositórios privados e públicos, sendo o pago e gratuito, respectivamente.
No GitHub, é possível, dentre tantas outras funcionalidades, visualizar o código no navegador, criar \textit{issues}, revisar e aceitar mudanças, baixar o repositório como \textit{.zip}, criar \textit{branchs} ou até mesmo fazer o \textit{fork} de outro repositório.

\subsection{Waffle}
O Waffle\footnote{www.waffle.io} é uma solução de gerenciamento de projeto \textit{online}, que se integra ao repositório do GitHub e cria um \textit{board}, muito semelhante a um \textit{kanban}, com as \textit{issues} que estão cadastradas, e as agrupa de acordo com seus status, que pode ser Backlog, Ready, In Progress  e Done. Cada \textit{issue}, se estiver associada a um Milestone no GitHub, será também associada a um Milestone no Waffle. Qualquer alteração feita no GitHub é automaticamente compatibilizada no Waffle.

\subsection{LaTeX}
O Latex\footnote{http://www.latex-project.org/} é um sistema de preparação de documentos para composição tipográfica de alta qualidade, desenvolvido inicialmente por Leslie Lamport, em 1985, baseado na linguagem TeX criada por Donald E. Knuth, no final da década de 70, na Universidade de Stanford. Atualmente, o Latex é mantido e desenvolvido pelo \textit{The Latex3 Project}\footnote{http://latex-project.org/latex3.html}, sendo disponibilizado gratuitamente.
O Latex é mais amplamente utilizado no meio técnico-científico para escrita de documentos de médio a grande porte, devido a sua facilidade de produzir fórmulas e símbolos matemáticos. 

\subsection{TexMaker}
O TeXMaker\footnote{http://www.xm1math.net/texmaker/} é um editor de texto Latex gratuito, multiplataforma para Linux e MacOSX, com suporte a unicode, verificação ortográfica, auto-completar e contém um visualizador de pdf embutido.
Com uma interface simples, o TexMaker contém botões de atalho para as funções mais utilizadas como: estruturas (part/chapter/section), referências (ref/cite), tamanho e estilo de letra, negrito, itálico, alinhar (direta/esquerda, centro) e inserir diversos símbolos matemáticos.

\section{Considerações finais}